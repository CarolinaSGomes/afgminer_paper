\documentclass{acm_proc_article-sp}

\usepackage[usenames]{color}
\usepackage[latin1]{inputenc}
\usepackage{amsmath}
\usepackage{amsfonts}
\usepackage{amssymb}
\usepackage{epsfig}

\newif\ifComments
\Commentstrue
\newcommand{\REM}[1]{} 
\ifComments
\newcommand{\todo}[1]{\noindent\textcolor{red}{Todo: {#1}}}
\newcommand{\jna}[1]{\noindent\textcolor{blue}{Nelson: {#1}}}
\newcommand{\carol}[1]{\noindent\textcolor{green}{Carol: {#1}}}
\newcommand{\js}[1]{\noindent\textcolor{magenta}{Joerg: {#1}}}
\else
\newcommand{\todo}[1]{}
\newcommand{\jna}[1]{}
\newcommand{\carol}[1]{}
\newcommand{\js}[1]{}
\fi

\def\eg{{\it e.g., \/}}
\def\ie{{\it i.e., \/}}
\def\etal{{\it et al.\/}}
\def\viz{{\it viz.,\/ }}

\begin{document}

\title{Heavyweight Pattern Mining in Attributed Flow Graphs} 
\titlenote{This research is partially funded by a grant from the Natural Science and Engineering Research Council of Canada through a Collaborative Research and Development grant and by the IBM Centre for Advanced Studies}
\
\numberofauthors{3} 
\author{
\alignauthor
Carolina Sim\~oes Gomes\\
       \affaddr{Intuit Canada}\\
       \affaddr{Edmonton, AB, Canada}
% 2nd. author
\alignauthor
Jos\'e Nelson Amaral\\
       \affaddr{Department of Computing Science}\\
       \affaddr{University of Alberta}\\
       \affaddr{Edmonton, AB, Canada}
% 3rd. author
\alignauthor
Joerg Sander\\
       \affaddr{Department of Computing Science}\\
       \affaddr{University of Alberta}\\
       \affaddr{Edmonton, AB, Canada}
\and
\alignauthor 
Joran Siu\\
       \affaddr{IBM Toronto Software Laboratory}\\
       \affaddr{Markham, ON, Canada}
\alignauthor 
Li Ding\\
       \affaddr{Amazon Canada}\\
       \affaddr{Toronto, ON, Canada}
% \and  % use '\and' if you need 'another row' of author names
% 4th. author
}

\maketitle
\begin{abstract}
This paper defines a new problem - heavyweight pattern mining in attributed flow graphs. The problem can be described as the discovery of patterns in flow graphs that have sets of attributes associated with their nodes. A connection between nodes is represented as a directed edge. The amount of load that goes through a path between nodes, or the frequency of transmission of such load between nodes, is represented as edge weights. A heavyweight pattern is a sub-set of attributes, found in a dataset of attributed flow graphs, that are connected by edges and have a computed weight higher than an user-defined threshold. A new algorithm called AFGMiner is introduced, the first one to our knowledge that finds heavyweight patterns in a dataset of attributed flow graphs and associates each pattern with its occurrences. The paper also describes a new tool for compiler engineers, HEPMiner, that applies the AFGMiner algorithm to Profile-based Program Analysis modeled as a heavyweight pattern mining problem.
\end{abstract}

% A category with the (minimum) three required fields
\category{H.2.8}{Information Systems}{Database Management}[Database Applications,Data Mining]
%A category including the fourth, optional field follows...
\category{I.2.8}{Computing Methodologies}{Artificial Intelligence}[Problem Solving, Control Methods, and Search, Graph and tree search strategies]


%\terms{Data Mining}

%\keywords{ACM proceedings, \LaTeX, text tagging} % NOT required for Proceedings

\section{Introduction}
	\label{sec:intro}
	Flow graphs are an abstraction used to represent elements (\eg digital data, goods, electric current) that travel through a network of nodes (\eg computers, physical locations, circuit parts). Flow graphs are often used in the modelling of logistics problems. An attributed flow graph (AFG) is a single-entry/single-exit graph with a \emph{source node} that has no incoming edges and a \emph{sink node} with no outgoing edges.The flow starts in the source node and is directed to the sink node. All other nodes in the AFG, if they exist, must have at least one incoming and one outgoing edge. In addition, AFGs have attributes in their nodes representing, \emph{e.g.}, types of goods at a given network node, and weights in nodes and edges representing, \emph{e.g.}, the amount of goods stored at that node and flowing between any two nodes. An example of an AFG is shown in Figure~\ref{fig:AFG}.

\begin{figure}[htp]
\centering
\begin{tabular}{cc}
 %   \includegraphics[scale=0.2]{figures/attributed_flow_graph2.eps}
  \includegraphics[scale=0.15]{figures/attributed_flow_graph_descpr.pdf} &
  \includegraphics[scale=0.15]{figures/attributed_flow_graph3.pdf}\\
  (a) Elements of an AFG & (b) Example
\end{tabular}
    \caption{Example of attributed flow graph.}
    \label{fig:AFG}  
\end{figure}

Existing graph-mining algorithms are all severely limited in their applicability to AFGs. None of them is able to find general sub-graph patterns in AFGs that take into account multiple node attributes as well as weights in nodes and edges. The only work on mining patterns in AFGs is the FlowGSP algorithm proposed by \emph{Jocksch~\etal}~\cite{FlowGSP}. However, FlowGSP can only find sub-path patterns, while the algorithm described in this work finds all the patterns that FlowGSP finds and also finds additional patterns that encompass multiple sub-paths in the dataset.  

This paper presents AFGMiner, the first algorithm, to the best of our knowledge, to address the problem of mining AFGs for general sub-graph patterns. AFGMiner takes as input a set of AFGs and a support measure, ${\mathit MinSup}$, that takes into account the attribute weights, node weights and edge weights of the occurrences of patterns. AFGMiner returns all patterns $P$, called \emph{heavyweight patterns}, whose support ${\mathit MinSup}(P)$ is higher than a threshold. This threshold is user-specified as commonly assumed in pattern-mining approaches.
The main contributions of this paper are as follows: 

\begin{enumerate}
\item Definition of the Attributed-Flow-Graph Mining problem to find Heavyweight Patterns.

\item Development of different versions of AFGMiner, an algorithm that mines for heavyweight patterns in attributed flow graphs, including a parallel version with a work-distribution heuristic to maintain workload balance between multiple threads.

\item Development of HEPMiner, a tool that automates the analysis of hardware-instrumented profiles, as an application of AFGMiner. Patterns discovered by HEPMiner indicate potential, non-obvious, opportunities for compiler and architecture-design improvements.

\item Complexity and performance analysis of AFGMiner, comparison against the FlowGSP algorithm and qualitative analysis of patterns found by HEPMiner when applied to the DayTrader benchmark running on IBM's WebSphere Application Server~\cite{WAS}.
\end{enumerate} 






\section{Problem Definition}
        \label{sec:ProblemDef}
        \input{text/ProblemDefinition.tex}
%\section{The Profiled-Based Program\\ 
%              Analysis Problem}
%	\label{sec:ProgramAnalysis}
%	\begin{figure}[h!]
\centering
    \includegraphics[scale=0.3]{figures/plot1.eps}
    \caption{Plot 1 description.}
    \label{fig:Plot1}  
\end{figure}

\begin{figure}[h!]
\centering
    \includegraphics[scale=0.3]{figures/plot2.eps}
    \caption{Plot 2 description.}
    \label{fig:Plot2}  
\end{figure}


\section{The AFGMiner Algorithm}
	\label{sec:AFGMiner}
	AFGMiner generates and tests candidate patterns of increasing number of edges, then extends those patterns considered heavyweight. It generates candidate patterns of $k$ edges, starting with $k = 0$ (patterns composed of a single node and no edges), and searches for occurrences of such patterns in the dataset by using a sub-graph isomorphism detection algorithm, modified to take attributes into consideration~\ref{ThesisRef}. The prototype implementation of AFGMiner adapts VF2~\cite{Cordella}, an algorithm that is faster than alternatives for graphs that are relatively regular, have a large number of nodes and whose nodes have small valency~\cite{Foggia}, such as the ones found in the case study. Each ocurrence found has its node weight and edge weight support values computed, and, when no more occurrences of a pattern are present in the dataset, the support value for the pattern itself is computed by aggregating the support values of its occurrences. If the support value for the pattern is higher than an user-defined threshold, the pattern is heavyweight. It is then output to the user and later extended into candidate patterns with an additional edge, a process called \emph{edge-by-edge pattern extension}. If the pattern is not heavyweight, it is discarded. Edge-by-edge pattern extension works by adding to a pattern either: (i) an edge that connects two of its nodes; (ii) or an edge that connects one of its nodes to a new node called the \emph{extension node}. A pattern that generates other patterns by extension is called a \emph{parent pattern}, while the generated patterns are \emph{child patterns}.

\subsection{Canonical Labeling}
Two different patterns, when extended, may generate child patterns that are isomorphic, so redundancy should be detected using the well-known concept of \emph{canonical labelling}~\cite{gSpan}. The idea is to map each sub-graph pattern to an identifier string called \emph{DFS Code} after labelling its nodes and edges. DFS Codes can then be lexically ordered in such a way that, if two sub-graphs are isomorphic to each other, they provably have the same minimum DFS Code. The rules that define how to sort DFS Codes depend on the types of graphs being mined.~\ref{ThesisRef}

\subsection{Support Value Policy}
AFGMiner adopts an \emph{anti-monotonic} support value policy to enable the pruning of its search space. Under this policy, the support value of a pattern is always lower than or equal to the support value of any of its ancestor patterns. Therefore, if a pattern is not heavyweight, none of its descendants can be heavyweight, such that all patterns that do not meet a minimum support criteria can be discarded. The support value policy works as follows. For each occurrence $g$ of a pattern $p$, two values are calculated: $S_n(g)$, the weight of the attribute with minimum weight amongst all attributes associated with nodes of $g$; and $S_e(g)$, the minimum edge weight amongst all edges of $g$. The $S_n(g)$ of all occurrences of $p$ found in $DS$ are then added up, resulting in $S_n(p)$, the node weight support of $p$. The $S_e(g)$ of all occurrences of $p$ found in $DS$ are also added up, resulting in $S_e(p)$, the edge weight support of $p$. The support value for $p$ is $S_m(p)$, the maximum between $S_n(p)$ and $S_e(p)$. $S_m(p)$ is compared against the support threshold to decide whether $p$ is a heavyweight pattern. The support-value policy selected for AFGMiner is anti-monotonic because only the minimum edge weight and the minimum node attribute weight of each occurrence are used in the computation.

\subsection{Generation of Candidate Patterns}
AFGMiner mines for candidate patterns with an increasing number of attributes in their single node (in the case of 0-edge patterns) or in their extension node (in the case of $k$-edge patterns with $k > 0$), starting with a single attribute, and then combining attributes such that an attribute set is used to generate a pattern only if all of its sub-sets generated heavyweight patterns. This process is called \emph{attribute-set growth}. A set of patterns that have the same number of edges is called a \emph{generation}. Patterns of a certain generation are only created and mined after all the heavyweight patterns of the previous generation have been found. This is important because it allows the algorithm to use only the $A_k$ set of distinct attributes present in the $(k - 1)$-th generation to compose the patterns of the $k$-th generation, thus restricting the number of candidate patterns produced.

%\begin{figure}[h!]
%\centering
 %   \includegraphics[scale=0.15]{figures/example_0-edge.pdf}
   % \caption{0-edge candidate patterns.}
    %\label{fig:example_0-edge}  
%\end{figure}

%\begin{figure}[h!]
%\centering
   % \includegraphics[scale=0.15]{figures/example_0-edge_2-attribute.pdf}
    %\caption{Attribute-set growth for 0-edge candidate patterns.}
    %\label{fig:example_0-edge_2-attribute}  
%\end{figure}

%\begin{figure}[h!]
%\centering
   % \includegraphics[scale=0.15]{figures/example_1-edge.pdf}
    %\caption{1-edge candidate patterns.}
    %\label{fig:example_1-edge}  
%\end{figure}

%\begin{figure}[h!]
%\centering
  %  \includegraphics[scale=0.15]{figures/example_1-edge_2-attribute.pdf}
   % \caption{Attribute-set growth for 1-edge candidate pattern node $a$ $\rightarrow$ $a$. }
    %\label{fig:example_1-edge_2-attribute}  
%\end{figure}

%\begin{figure}[h!]
%\centering
 %   \includegraphics[scale=0.15]{figures/example_2-edge.pdf}
  %  \caption{2-edge candidate patterns.}
   % \label{fig:example_2-edge}  
%\end{figure}



\section{Algorithm Improvements}
	\label{sec:Improvements}
	The previous section described the original version of AFGMiner, called {\bf AFGMiner-iso} (\emph{iso} stands for isomorphism detection). AFGMiner-iso visits all nodes in the dataset for every pattern searched, making it potentially slow when analyzing large datasets composed of thousands of AFGs with hundreds of nodes each, as in the case study presented in this work. Another version of AFGMiner, {\bf AFGMiner-locreg}, addresses this performance issue. It uses the concept of location registration.

{\bf Location Registration} is the complete mapping, including nodes and edges, between a candidate pattern $p$ and each of its occurrences $g$. If $p$ is found to be heavyweight, this mapping is kept, otherwise it is discarded. Then, when a heavyweight pattern $p$ is extended into its child patterns $c$, in order to find occurrences of $c$ the algorithm only checks the mappings between $p$ and its occurrences $g$. In order to generate $c$, $p$ has one of its nodes, $v$, extended by adding to it an edge $e$, and may also have an extension node $q$ connected to $e$. The idea of location registration is to check each mapping between $p$ and occurrences $g$ for: (i) the node $v_g$ in $g$ that corresponds to $v$; (ii) if $v_g$ is connected to an edge $e_g$ that corresponds to $e$ and (iii) in case $c$ was extended from $p$ by adding an extension node, check if $e_g$ connects a node $q_g$, corresponding to node $q$, to $v_g$. If the algorithm is able to find appropriate $v_g$, $e_g$ and $q_g$ attached to the occurrence $g$, then the sub-graph that is composed of $g$ plus $e_g$ and $q_g$ is an occurrence of $c$.







        \subsection{A Parallel Implementation of\\ 
              AFGMiner}
	\label{sec:pAFGMiner}
	A parallel version of AFGMiner benefits from the multiple cores available in many computing systems. An important challenge in the implementation of a parallel version of AFGMiner-locreg, {\bf p-AFGMiner}, is the distribution of the workload to improve load balancing.

p-AFGMiner executes the following steps while a queue $Q$ of heavyweight patterns is not empty (s signals a sequential step, while p signals a parallel step): (i-s) fork into $n$ threads to start the processing of a new generation composed of patterns with $k$ edges.

For $k = 0$, (ii-s) divide the set $A_0$ among the threads and (iii-p) each thread generates 0-edge candidate patterns using their part of $A_0$ and searches the database for these patterns --- heavyweight patterns form a local queue $Q_{TL}$ and the set of attributes in $Q_{TL}$ form $A_{TL}$, the local set of attributes.

For $k > 0$, (ii-s) distribute the queue $Q$ of $k$-edge heavyweight patterns among $n$ thread-local queues $Q_{TL}$; (iii-p) each thread generates $(k+1)$-edge candidates from patterns in $Q_{TL}$ and computes their support, generating $A_{TL}$ and adding the patterns found to be heavyweight to $Q_{TL}$; 

Then: (iv-s) synchronize when all threads finish step (iii) and unify all $Q_{TL}$s into a single queue $Q$, and all $A_{TL}$s into a single $A_{k+1}$ set of attributes; and (v) start processing the next generation, if $Q$ is not empty. The dataset of AFGs, $DS$, is read-only during the entire run of p-AFGMiner, allowing the maintenance of thread-local versions of variables and thus reducing synchronization.

The distribution of $Q$ among threads should balance the workload to improve the performance of p-AFGMiner. The number of occurrences of the parent of a pattern $p$ in $DS$ is the most important factor determining the time required to search for occurrences of $p$. A reasonable heuristic tries to balance the number of parent-pattern occurrences assigned to each thread. 

\begin{figure}[h!]
\centering
    \includegraphics[scale=0.3]{figures/WorkDistributionHeuristic.pdf}
    \caption{Illustration of pattern distribution heuristic.}
    \label{fig:WorkDistributionHeuristic}  
\end{figure}

The pattern-distribution heuristic created for p-AFGMiner sorts the $m$ patterns in $Q$ by decreasing order of the number of parent-pattern occurrences. The heuristic then does a round-robin assignment of patterns to threads following an increasing order for the patterns with an even position in the sorted $Q$ and in decreasing order for patterns with an odd position in the sorted $Q$ as illustrated in Figure~\ref{fig:WorkDistributionHeuristic} --- assume that numbering in $Q$ starts at zero. In this illustration each ellipse represents a pattern in $Q$ and the size of the ellipse stands for the number of occurrences of the pattern's parent in the database.
This simple $O(n)$ heuristic is effective for a moderate number of threads and for limited variations in the number of parent-pattern occurrences. Experimental evaluation revealed that this workload-distribution heuristic lowered the execution time of p-AFGMiner, on average, by 6\% when compared with a naive workload distribution method that simply distributes patterns among the threads without any sorting.


 


 

	
\section{Case Study: Using AFGMiner\\
              for Program Analysis}
	\label{sec:CaseStudies}
	The process of optimizing computer systems, compilers, computer architecture and computer applications often involves the analysis of the dynamic behaviour of an application. When this optimization is performed offline it involves the analysis of the runtime profile collected over a single, or many, executions of an application. For many applications it is possible to identify {\em hot-spots} in the application profile that consist of segments of the application's source code that account for a larger portion of the execution time. The effort to improve any part of the computer system that influences the application performance can then easily focus on these hot-spots. However, there is a class of computer applications that have no such hot-spots. Instead, the execution time is distributed over a very large code base, with no method taking up signicant execution time (\ie no more than 2 or 3\%). Such applications are said to have \emph{flat profiles}.

The execution of such applications also generates very large profiles that are difficult to analyze by manual inspection. Thus, the {\bf Profile-based Program Analysis} (PBPA) problem is defined as follows. Given a profile {\em Prof} obtained from an execution of a computer program, automatically discover operation patterns in the execution of {\em Prof} that, in aggregation, account for a sufficiently large fraction of the program's execution time. Developers are then able to focus their optimization efforts on those areas in the program code that correspond to occurrences of the relevant operation patterns. In order to solve PBPA, we convert it to a heavyweight pattern mining problem, by modeling the program as a dataset of attributed flow graphs named \emph{Execution Flow Graphs} (EFGs). The EFG is an attributed flow graph that has nodes, edges, weights and attributes whose semantics vary according to the target user of the program analysis. AFGMiner is then applied to the dataset of EFGs, and heavyweight patterns found are the ones relevant to developers. 

This section discusses the run-time complexity of AFGMiner when mining EFGs. In general, the complexity of AFGMiner depends on the type of attributed flow graph being mined.

The section also presents two practical applications of AFGMiner to PBPA. The first is HEPMiner, a tool that targets compiler and architecture developers and intends to facilitate the analysis of programs that have been profiled by hardware instrumentation\cite{Nagpurkar-caecw07}. 

The second practical application of AFGMiner is SCPMiner, a tool that targets application developers and associates heavyweight patterns with source-code lines that the developer may modify in order to obtain performance improvements.

\subsection{Sub-Graph Mining in Bounded Treewidth Graphs}

A structured program is one that combines sub-programs to compute a function, and does so by using any combination of three fundamental control structures: (i) sequential execution of sub-programs; (ii) selective execution of certain sub-programs instead of others by evaluating boolean variables; and (iii) execution of a sub-program repeatedly until a boolean variable is true. More specifically, a structured program is free of \emph{goto}-clauses. The classic work of \emph{Bohm and Jacopini} shows that any program using \emph{goto} can be converted into a \emph{goto}-free form~\cite{Bohm}. This conversion is done by (\eg C and C++) compilers as part of the transformation from source-code to a compiler-specific and language-agnostic Intermediate Representation (IR). \emph{Thorup} shows that graphs representing the control flow of structured programs (\ie Control Flow Graphs) have tree-width of at most six~\cite{Thorup}. Therefore, CFGs of structured programs have small tree-width, or, more generally, bounded tree-width. 

Tree-width is a measure of how similar to a tree a graph is. It is a very useful property because several NP-hard problems on graphs become tractable for the class of graphs with bounded tree-width, including sub-graph isomorphism detection and, as a consequence, frequent sub-graph mining of connected graphs~\cite{Horvath}. \emph{Horvarth and Ramon} discovered a level-wise sub-graph mining algorithm that lists frequent connected sub-graphs in incremental polynomial time in cases when the tree-width of the graphs being mined is bounded by a constant~\cite{Horvath}. Because CFGs have bounded tree-width, so do the AFGs that represent CFGs, \ie EFGs. In the applications of AFGMiner shown in this paper, AFGMiner runs in incremental polynomial time because the problem being solved is fundamentally the problem of finding frequent connected sub-graphs in a dataset of bounded tree-width flow graphs. The addition of weights in nodes and edges and weighted attributes to nodes does not change the complexity of the algorithm. 

The fact that nodes, edges and attributes are weighted and edges are directed does not interfere in the complexity of the algorithm, but the generation of attribute sets of increasing size when creating new candidate patterns does. However, the number of attributes that an extension node of a $k$-edge candidate pattern with $k > 0$ or that the single node of a 0-edge candidate pattern can have is bounded by the size of $A$, \ie by the number of possible attributes that each pattern node may contain. As a consequence, the attribute-set growth component of the algorithm has a constant complexity, while the sub-graph mining component has incremental polynomial complexity. We can thus say that AFGMiner has incremental polynomial complexity when applied to PBPA.



	\subsection{A Tool for Compiler Developers\\
	                   and Computer Architects}
		\label{sec:HEPMiner}
		HEPMiner is a program performance analysis tool that requires information about the static control flow of the program and its behavior at run-time. Dynamic information about the program is obtained by profiling its methods and includes which instructions are executed, which hardware events are triggered as a result of instruction execution and the execution frequency of instructions and events measured in CPU cycles. HEPMiner assembles one EFG per profiled method. An EFG node is created for each assembly instruction in the profile, edges are created according to edges connecting basic blocks and the frequency of such edges is set to the frequency of the corresponding CFG edges. The weight of each EFG node is the number of sampling ticks associated with the corresponding assembly instruction that the node represents, and the attributes of a node are the events associated with the same instruction.  With all EFGs assembled, HEPMiner uses the AFGMiner algorithm to find \emph{heavyweight execution patterns} (HEP). HEPs are sets of hardware-related events captured by performance counters and associated with assembly instructions that were executing when the hardware-related events happened. Examples of such events are the occurrence of instruction and data cache misses, address-generation interlocks. These events affect program performance and it is thus important for compiler and architecture developers to understand when and why they happen, and which sets of events, correlated with which sets of instructions, take up more execution time. HEPMiner automates this process, and finds non-obvious correlations represented by execution patterns that take up significant execution time when their occurrences are considered in aggregation.

		
\section{Performance Evaluation\\
              Methodology}
	\label{sec:Methodology}
	In the experimental evaluation of AFGMiner, the tools HEPMiner and SCPMiner were separately tested in order to analyze the patterns that AFGMiner is able to uncover and how scalable the algorithm is in terms of run-time. Experiments for HEPMiner were performed on an Intel Core 2 Quad CPU Q6600 machine, running at 2.4 GHz and with 3 GB of RAM. The operating system installed in the machine was a Microsoft Windows XP Professional Edition, Service Pack 3. Experiments for SCPMiner ran on an AMD Athlon II Neo K125 running at 1.70 GHz and with 4 GB of RAM, of which 3.75 GB were usable. The operating system installed in the machine was a 64-bit Windows 7 Home Premium, Service Pack 1.

\subsection{HEPMiner Evaluation}
The experimental evaluation of the algorithm in the context of HEPMiner used profiles from the DayTrader Benchmark, running on WebSphere Application Server, on a z196 mainframe and JIT-compiled using the IBM Testarossa JIT compiler. The WebSphere Application Server is a Java\textregistered~Enterprise Edition (JEE) server developed by IBM and written in Java~\cite{WAS}. It has a very flat profile, with its execution time spread relatively evenly over 2,566 methods, as is typical of large business applications. The IBM System z is a generation of mainframe products by IBM~\cite{zEnterprise}. The IBM zEnterprise System 196, or z196 model, is a CISC mainframe architecture with a superscalar, out-of-order pipeline.

AFGMiner-locreg, as used by HEPMiner, had its run-time scalability tested in two experiments, A and B, by running the algorithm on DayTrader methods. Three important parameters that were taken into account in this analysis are the \emph{Minimum Hotness Method} (MMH) value, the minimum support threshold (MinSup) and the maximum allowed size of the attribute set in each candidate pattern node (MaxAttrs). The MMH is calculated by dividing the sum of all CPU cycles associated with each one of DayTrader's profiled methods by the cycles associated with the entire program run. For experiments {\bf A}and {\bf B}, the MMH was kept at 0.001, meaning that only methods with execution time that exceeds 0.1\% of program run-time are selected for mining (\emph{i.e.} in DayTrader, 278 of 2,566).

\emph{Experiment {\bf A}} compared the run-times of AFGMiner-locreg with respect to changes in MaxAttrs. MinSup was kept at 0.001, meaning that only those patterns consuming more than 0.1\% of program run-time are considered heavyweight. The goal was to measure the impact on run-time of attribute set sizes in candidate pattern nodes. The higher the number of attributes allowed, the more candidate patterns can be potentially generated, which is why modifying this parameter is a way of controlling the memory consumed by the algorithm and, we conjectured, also its run-time. \emph{Experiment {\bf B}} compared the run-times of AFGMiner-locreg with respect to changes in the number of EFG nodes visited by the algorithm, by changing MinSup but keeping MaxAttrs at 5.

Patterns found by AFGMiner-locreg were also compared to the ones found by the FlowGSP algorithm, which finds only sub-path patterns in AFGs. FlowGSP was modified to support variations in MMH, MinSup and MaxAttrs, and run on DayTrader methods with the same parameter values used for Experiment {\bf B} above. In addition, an expert compiler engineer from IBM's JIT Compiler Development team verified the usefulness of patterns identified by the HEPMiner tool, as described in Section~\ref{sec:QualAnalysis}.

\subsection{SCPMiner Evaluation}
The experimental evaluation of SCPMiner focused on verifying if the patterns found by the tool are useful to developers (Section~\ref{sec:QualAnalysis}). The benchmarks \emph{bzip2}, \emph{gobmk}, \emph{mcf} and \emph{namd}, all from the SPEC CPU2006~\cite{SPEC} suite, were analyzed by an expert developer from IBM's Multi-core Performance Tooling team, using the source-code of the benchmarks and previous knowledge about them. He compared source-code lines associated with heavyweight pattern occurrences, pointed by SCPMiner, to source lines indicated by an internal IBM tool being developed by the team. This internal tool highlights the source-code lines that consume the most CPU cycles when a program is profiled by \emph{tprof}. The IBM tool is thus only useful when the profile of analyzed applications is not flat. As a consequence, for purposes of validation, the four benchmarks we analyzed do not have flat profiles, so that the results of SCPMiner and the IBM tool could be more easily compared.









\section{Performance Evaluation}
	\label{sec:PerformanceEvaluation}
	In the experiments using HEPMiner, AFGMiner-locreg was always faster than AFGMiner-iso. As an example, when running with MMH and MinSup of 0.001, AFGMiner-iso took approximately 11 hours to complete the mining process, while AFGMiner-locreg took 40 minutes and p-AFGMiner with 8 threads took 15 minutes. The dramatic decrease in run-time when comparing AFGMiner-locreg to AFGMiner-iso is expected because location registration decreases the number of dataset sub-graphs that must be tested for isomorphism with candidate patterns. 

We also compared patterns found by AFGMiner and FlowGSP. Figure~\ref{fig:NumPatterns} shows the number of output patterns for different MinSup values. As expected, AFGMiner found all patterns found by FlowGSP, but also found additional patterns composed of multiple sub-paths. For flat profiles, the trend is for AFGMiner to find many more patterns than FlowGSP as the MinSup is decreased, with the sequential patterns found by both being the parent patterns of the sub-graph patterns found exclusively by AFGMiner. 

\begin{figure}[h!]
\centering
    \includegraphics[scale=0.18]{figures/numPatternsComp.pdf}
    \caption{Patterns found by FlowGSP and AFGMiner}
    \label{fig:NumPatterns}
\end{figure}

Relevant observations from the experiments are:

\begin{enumerate}
\item The MaxAttrs value is not as relevant a factor in the run-time performance of AFGMiner as the number of EFG nodes visited during the mining process.
\item The run-time of AFGMiner increases moderately with the number of EFG nodes visited.
\item Although increasing the number of threads logically decreases run-time, there is a diminishing effect as MinSup/MMH increase, as time spent on data loading and temporary bookkeeping of patterns and pattern occurrences - both done serially by p-AFGMiner - start dominating the total run-time.
\end{enumerate}

\begin{figure}[h!]
\centering
    \includegraphics[scale=0.4]{figures/plot2.pdf}
    \caption{Experiment A.}
    \label{fig:Plot2}  
\end{figure}

\begin{figure}[h!]
\centering
    \includegraphics[scale=0.4]{figures/plot1.pdf}
    \caption{Experiment B.}
    \label{fig:Plot1}  
\end{figure}

\begin{figure}[h!]
\centering
    \includegraphics[scale=0.2]{figures/experimentC.pdf}
    \caption{Experiment C.}
    \label{fig:Plot3}  
\end{figure}

\begin{figure}[h!]
\centering
    \includegraphics[scale=0.2]{figures/experimentD.pdf}
    \caption{Experiment D.}
    \label{fig:Plot4}  
\end{figure}

\section{Qualitative Analysis}
        \label{sec:QualAnalysis}
        This section discusses the patterns obtained by both HEPMiner. Results were analyzed by compiler engineers from the IBM Canada Software Laboratory. They found HEPMiner to be a useful tool and were able to not only validade results according to previous knowledge about the benchmark, but also to make new observations about the run-time behavior of DayTrader when executed by the z196 hardware. The observations are summarized below.
\begin{enumerate}
\item AFGMiner found heavyweight patterns that contain an edge from a branch instruction leading to a node that has instruction cache misses as one of its attributes. Reviewing the occurrences of such patterns shows that this edge represents the taken path from the branch, which confirms expectation that instruction cache misses should be observed only on the taken branch target.
\item From prior analysis, the experts at IBM know that 30\% of overall CPU cycles in JITted code are assigned to method prologues, and 40\% of instruction cache misses are correlated with method prologues. This was confirmed in the results output by AFGMiner.
\item An attribute that represents a non-taken, correct-direction branch prediction was found to be dominant by AFGMiner. The fact that this attribute shows up highlights that the JIT compiler performs well when ordering basic blocks to optimize for fall-through paths. This discovery led to the creation of performance counters that take into account the ratio of taken and non-taken branches.
\item The relative support values for a sequence of three attributes, present in several patterns, were found to be relevant by the IBM developers. The attributes are an address-generation interlock, followed by a directory or data cache miss, and then an instruction used to load a compressed referenced field from an object. This discovery led to the implementation of the pattern in the compiler's instruction scheduler in order to reduce its negative effect on the run-time of future compiled applications.
\end{enumerate}
\section{Related Work}
	\label{sec:Related}
	FlowGSP is a sequential-pattern mining algorithm that finds patterns whose occurrences are sub-paths of AFGs~\cite{JockschCC10, FlowGSP}. AFGMiner differs from FlowGSP in that it is able to find not only sequential patterns, but also patterns whose occurrences are sub-graphs of AFGs. AFGMiner takes less time than FlowGSP to find each pattern. In addition, AFGMiner is able to map each pattern to all its occurrences and output this mapping to the user.

gSpan is a classic sub-graph mining algorithm. The main difference between AFGMiner and gSpan is that AFGMiner is able to handle multiple node attributes and uses breadth-first search with eager pruning when generating candidate patterns, while gSpan follows a depth-first approach~\cite{gSpan}. \emph{Fast Frequent Subgraph Mining} (FFSM) is another well-known sub-graph mining algorithm for undirected graphs, and its novelty was the introduction of embeddings that make the mining process faster. AFGMiner-locreg also uses embeddings, but has to record the complete mapping between sub-graph patterns and occurrences, making them more memory-consuming. Gaston is a more recent frequent path, tree and graph miner integrated into a single algorithm. In contrast to AFGMiner, however, Gaston only mines for patterns in non-attributed, undirected and unweighted graphs~\cite{Gaston}. \emph{Horvart \etal} describes a sub-graph mining algorithm for outerplanar graphs~\cite{HorvathKDD06}. Similarly to AFGMiner the algorithm runs in incremental polynomial time due to outerplanar graphs being similar enough to trees.

A work related to HEPMiner is that of \emph{Dreweke \etal}. It uses sub-graph mining as part of a code transformation to extract code segments~\cite{DrewekeCGO07}. HEPMiner differs from this work in that it is an external performance analysis tool that helps compiler developers to reach conclusions about the performance of applications of interest, and detect improvement opportunities. 


\section*{Conclusion}
	\label{sec:Conclusion}
	This work defined heavyweight patterns and the problem of Heavyweight Pattern Mining. It presented AFGMiner, a heavyweight pattern mining algorithm that is generic enough to be applied to any problem that requires mining of attributed flow graphs, and is able
to find both sub-path and sub-graph patterns. AFGMiner was improved from its original version to use the concepts of location registration. In addition, a parallel version of AFGMiner with location registration was developed that uses a workload distribution heuristic to better balance the mining work performed by different threads.

The tools HEPMiner and SCPMiner, targeted at compiler and application developers, respectively, were created as useful applications of AFGMiner. The heavyweight patterns obtained by such tools were positively evaluated by experts from the IBM Canada Software Laboratory and gave them useful insight into the run-time behavior of analyzed programs.



\section{Acknowledgments}
This research is partially funded by a grant from the Natural Science and Engineering Research Council of Canada through a Collaborative Research and Development grant and by the IBM Centre for Advanced Studies

\bibliographystyle{abbrv}
\bibliography{local}  
%\appendix
%\section{First Abstract}
\balancecolumns
\end{document}
