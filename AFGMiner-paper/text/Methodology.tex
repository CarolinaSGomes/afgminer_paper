Experiments for HEPMiner were performed on an Intel Core 2 Quad CPU Q6600 machine, running at 2.4 GHz and with 3 GB of RAM.

\subsection{HEPMiner Evaluation}
The experimental evaluation of the algorithm in the context of HEPMiner uses profiles from the DayTrader Benchmark, running on IBM's WebSphere Application Server, on an IBM z196 mainframe~\cite{zEnterprise} and JIT-compiled using the IBM Testarossa JIT compiler. The WebSphere Application Server is a Java\textregistered~Enterprise Edition (JEE) server~\cite{WAS}. It has a very flat profile, with its execution time spread relatively evenly over 2,566 methods, as is typical of large business applications.

Four experiments examine the performance trends of AFGMiner: {\bf A}, {\bf B}, {\bf C} and {\bf D}, described in more detail below. All experiments were run on DayTrader methods that were profiled using hardware instrumentation. Three important parameters in the experiments are: the minimum support threshold (MinSup) and the maximum allowed size of the attribute set in each candidate pattern node (MaxAttrs), both described in Section~\ref{sec:ProblemDef}; and the \emph{Minimum Hotness Method} (MMH) value. The MMH is calculated by dividing the sum of all CPU cycles associated with each one of DayTrader's profiled methods by the cycles associated with the entire program run. This parameter is used in the experiments simply to limit the methods analyzed by the algorithm to those that contribute most significantly to total program run-time, and are thus more likely to contain patterns of interest to compiler engineers. 

For experiments {\bf A}, {\bf B} and {\bf C}, the MMH is kept at 0.001, meaning that only methods with execution time that exceeds 0.1\% of program run-time are selected for mining (\emph{i.e.} in DayTrader, 278 of 2,566). For experiment {\bf D}, the MMH has values 0.001 (278 methods), 0.003 (56 methods) and 0.005 (23 methods). For all experiments except {\bf A}, MaxAttrs is kept at 5.

\emph{Experiment {\bf A}} compares the run-times of AFGMiner-locreg with respect to changes in MaxAttrs. MinSup is kept at 0.001, meaning that only those patterns consuming more than 0.1\% of program run-time are considered heavyweight. Increasing MaxAttrs potentially causes more candidate patterns to be generated, which is why modifying this parameter is a way of controlling the memory consumed by the algorithm and also its run-time. 

\emph{Experiment {\bf B}} compares the run-times of AFGMiner-locreg with respect to changes in the number of EFG nodes visited by the algorithm. The number of EFG nodes visited is controlled by changing MinSup.

\emph{Experiment {\bf C}} compares run-times of AFGMiner-locreg and p-AFGMiner with 2, 4, 6 and 8 threads, by changing MinSup.

\emph{Experiment {\bf D}} compares run-times of AFGMiner-locreg and p-AFGMiner with 2, 4, 6 and 8 threads by changing MMH. MinSup is kept at 0.001.

Patterns found by AFGMiner-locreg were also compared to the ones found by the FlowGSP algorithm. FlowGSP was modified to support variations in MMH, MinSup and MaxAttrs, and run on DayTrader methods with the same parameter values used for Experiment {\bf B} above. In addition, expert compiler engineers from IBM's JIT Compiler Development team verified the usefulness of patterns identified by the HEPMiner tool, as described in Section~\ref{sec:QualAnalysis}.







