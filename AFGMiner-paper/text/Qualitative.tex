This section discusses the patterns obtained by both HEPMiner. Results were analyzed by expert developers from the IBM Canada Software Laboratory.
\subsection{HEPMiner Results}
The compiler engineers found HEPMiner to be a useful tool and were able to not only validade results according to previous knowledge about the benchmark, but also to make new observations about the run-time behavior of DayTrader when executed by the z196 hardware. The observations are summarized below.
\begin{enumerate}
\item AFGMiner found heavyweight patterns that contain an edge from a branch instruction leading to a node that has instruction cache misses as one of its attributes. Reviewing the occurrences of such patterns shows that this edge represents the taken path from the branch, which confirms expectation that instruction cache misses should be observed only on the taken branch target.
\item From prior analysis, the experts at IBM know that 30\% of overall CPU cycles in JITted code are assigned to method prologues, and 40\% of instruction cache misses are correlated with method prologues. This was confirmed in the results output by AFGMiner.
\item An attribute that represents a non-taken, correct-direction branch prediction was found to be dominant by AFGMiner. The fact that this attribute shows up highlights that the JIT compiler performs well when ordering basic blocks to optimize for fall-through paths. This discovery led to the creation of performance counters that take into account the ratio of taken and non-taken branches.
\item The output by AFGMiner shows patterns with low support value that have a certain attribute that is actually an instruction which is part of an asynchronous check sequence. This check sequence is used as a cooperative point in the method to allow for garbage collection and/or JIT compilation-related sampling mechanisms that determine which method is being executed. Such checks are placed at method entries and within hot loops. The experts at IBM confirmed that, given that DayTrader is a very flat benchmark, it is correct that the pattern should have such a low support.
\item The relative support values for a sequence of three attributes, present in several patterns, were found to be relevant by the IBM developers. The attributes are an address-generation interlock, followed by a directory or data cache miss, and then an instruction used to load a compressed referenced field from an object. This discovery led to the implementation of the pattern in the compiler's instruction scheduler in order to reduce its negative effect on the run-time of future compiled applications.
\end{enumerate}