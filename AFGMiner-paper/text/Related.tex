FlowGSP is a sequential-pattern mining algorithm that finds patterns whose occurrences are sub-paths of AFGs~\cite{JockschCC10, FlowGSP}. AFGMiner differs from FlowGSP in that it is able to find not only sequential patterns, but also patterns whose occurrences are sub-graphs of AFGs. AFGMiner takes less time than FlowGSP to find each pattern. In addition, AFGMiner is able to map each pattern to all its occurrences and output this mapping to the user.

gSpan is a classic sub-graph mining algorithm. The main difference between AFGMiner and gSpan is that AFGMiner is able to handle multiple node attributes and uses breadth-first search with eager pruning when generating candidate patterns, while gSpan follows a depth-first approach~\cite{gSpan}. \emph{Fast Frequent Subgraph Mining} (FFSM) is another well-known sub-graph mining algorithm for undirected graphs, and its novelty was the introduction of embeddings that make the mining process faster. AFGMiner-locreg also uses embeddings, but has to record the complete mapping between sub-graph patterns and occurrences, making them more memory-consuming. Gaston is a more recent frequent path, tree and graph miner integrated into a single algorithm. In contrast to AFGMiner, however, Gaston only mines for patterns in non-attributed, undirected and unweighted graphs~\cite{Gaston}. \emph{Horvart \etal} describes a sub-graph mining algorithm for outerplanar graphs~\cite{HorvathKDD06}. Similarly to AFGMiner the algorithm runs in incremental polynomial time due to outerplanar graphs being similar enough to trees.

A work related to HEPMiner is that of \emph{Dreweke \etal}. It uses sub-graph mining as part of a code transformation to extract code segments~\cite{DrewekeCGO07}. HEPMiner differs from this work in that it is an external performance analysis tool that helps compiler developers to reach conclusions about the performance of applications of interest, and detect improvement opportunities. 
