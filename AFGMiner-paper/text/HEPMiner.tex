HEPMiner is a program performance analysis tool that requires information about the static control flow of the program and its behavior at run-time. Dynamic information about the program is obtained by profiling its methods and includes which instructions are executed and which hardware events are triggered as a result of instruction execution. When the program is profiled, compiler log files (typically one per running thread) are created with edge profiling information, \ie basic blocks and flow edges of each method and their execution frequency measured in CPU cycles. A hardware profile is also generated and contains sequences of instructions and associated hardware events, captured by performance counters during run-time. Each instruction has a corresponding number of cycles during which it was executed, and the number of cycles in which each event was active. 

The information from compiler log files and hardware profile goes into database tables later read by HEPMiner, that assembles one EFG per profiled method. An EFG node is created for each assembly instruction in the profile; instructions in the same basic block are connected by new flow edges and the frequency of such edges is the same as the frequency of those basic blocks that they connect. The nodes at the end of each basic block are connected to the first nodes of all subsequent basic blocks as determined by edges in the CFG of the method, and the frequency of these edges is set to the frequency of the corresponding CFG edges. The weight of each EFG node is the number of sampling ticks associated with the corresponding assembly instruction that the node represents, and the attributes of a node are the events associated with the same instruction. 

With all EFGs assembled, HEPMiner uses the AFGMiner algorithm to find \emph{heavyweight execution patterns} (HEP). HEPs are sets of hardware-related events captured by performance counters and associated with assembly instructions that were executing when the hardware-related events happened. Examples of such events are the occurrence of instruction and data cache misses, address-generation interlocks. These events affect program performance and it is thus important for compiler and architecture developers to understand when and why they happen, and which sets of events, correlated with which sets of instructions, take up more execution time. HEPMiner automates this process, and finds non-obvious correlations represented by execution patterns that take up significant execution time when their occurrences are considered in aggregation.
