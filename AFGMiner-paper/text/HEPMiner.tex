HEPMiner is a program performance analysis tool that requires information about the static control flow of the program and its behavior at run-time. Dynamic information about the program is obtained by profiling its methods and includes which instructions are executed, which hardware events are triggered as a result of instruction execution and the execution frequency of instructions and events measured in CPU cycles. HEPMiner assembles one EFG per profiled method. An EFG node is created for each assembly instruction in the profile, edges are created according to edges connecting basic blocks and the frequency of such edges is set to the frequency of the corresponding CFG edges. The weight of each EFG node is the number of sampling ticks associated with the corresponding assembly instruction that the node represents, and the attributes of a node are the events associated with the same instruction.  With all EFGs assembled, HEPMiner uses the AFGMiner algorithm to find \emph{heavyweight execution patterns} (HEP). HEPs are sets of hardware-related events captured by performance counters and associated with assembly instructions that were executing when the hardware-related events happened. Examples of such events are the occurrence of instruction and data cache misses, address-generation interlocks. These events affect program performance and it is thus important for compiler and architecture developers to understand when and why they happen, and which sets of events, correlated with which sets of instructions, take up more execution time. HEPMiner automates this process, and finds non-obvious correlations represented by execution patterns that take up significant execution time when their occurrences are considered in aggregation.
